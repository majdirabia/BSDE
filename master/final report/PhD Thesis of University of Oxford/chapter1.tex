\chapter{Introduction}

\subsection{SDE}

In this section, we recall the basic tools from stochastic differential equations 
\begin{eqnarray}
dX_t = \mu(t,X_t)dt + \sigma (t, X_t) dB_t t\in [0,T]
\end{eqnarray}
where $T >0$ is a given maturity date.  Here,  $\mu$ and $\sigma$ are $\mathcal{F}_{t \geq 0 }\times \mathcal{B}(\mathbb{R^n})$ measurable functions from $[0,T] \times \Omega \times \mathbb{R^n}$ to $\mathbb{R^n}$ and $\mathcal{M}_{n, d}(\mathbb{R})$,  respectively. 

\begin{defn}
A strong  solution  of (1) is  an $\mathcal{F}_{t \geq 0 }$ measurable process $X_{t\geq 0}$ such 
that $\int_{0}^{T}(|\mu(t, X_t)|+|\sigma(t, X_t)|^2)dt < \infty \quad a.s.$  and 
\begin{eqnarray}
X_t = X_0 + \int_{0}^{T}(s, X_s)ds + \int_{0}^{T}\sigma(s, X_s)dW_s, t \in [0,T]
\end{eqnarray}
	
\end{defn}


\subsection{BSDE : problem, motivation and example}

\subsubsection{Problem}

All over this paper, we will be interested in the following so called backward stochastic differential equation (BSDE). 

\begin{eqnarray}
dX_t = \mu(t,X_t)dt + \sigma (t, X_t) dB_t\\
-dY_t = f(t,X_t, Y_t, Z_t)dt - Z_tdB_t \\
Y_T=\xi
\end{eqnarray}

In this representation, $(X_t)_{0 \leq t \leq T}$ is the forward p-dimensional process. 
\newline 
$(Y_t)_{0 \leq t \leq T}$ is the backward component, where f is the driver of the BSDE, $(B_t)_{0 \leq t \leq T}$ is a d-dimensional Brownian Motion defined on a filtered probability space $(\Omega, \mathcal{F}, \mathbb{P})$, $\xi$ a measurable function with respect to the filtration generated by the BM, and $(Y_t, Z_t)$ is a pair of square-integrable adapted processes satisfying the equation. 

We present here only a sketch of the paper we are working on. 
Most important results are detailed. 

\subsubsection{Motivation}

Thinking about pricing of an American Option leads to so called dynamic programming equation, requiring the computation of a conditional expectation, given current state. This derivative financial product gives a payoff at maturity $T$, so its pricing needs a backward process. 
The most famous algorithm for such pricing is the Longstaff-Schwartz, using a least square method for the regression step (see next chapter). Unfortunately, curse of dimensionality affects this method quickly (around ten assets), as it uses basis projection. 
Targeting more than ten-twenty assets, we decided to look at other methods, unusual in this case of study. 	

What is specific to American options compared to European options is its Partial Derivative Equation (PDE). The non-linearity makes it difficult to solve using a difference scheme method.
Most models in finance require solving a non-linear, or semi-linear PDE (CVA, passport option, ...). 

\subsubsection{Example}


\cite{Einstein}
