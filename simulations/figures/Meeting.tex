\documentclass[english,11pt,openany]{article}
\usepackage{graphicx}
\usepackage{ucs}
\usepackage[utf8]{inputenc}
\usepackage[executivepaper,margin=1in]{geometry}
%\usepackage[charter]{mathdesign}
\usepackage{babel}
\usepackage{subfigure}
\usepackage{fancyhdr}
\usepackage{listings}
\usepackage{lmodern}
\usepackage{amsmath}
\usepackage{amsthm}

\newtheorem{theorem}{Theorem}
\usepackage{amssymb}
\usepackage[colorlinks=true]{hyperref} 
\hypersetup{urlcolor=blue,linkcolor=black,citecolor=black,colorlinks=true}
\newtheorem{prop}{Proposition}
\usepackage[usenames,dvipsnames,svgnames,table]{xcolor}
\definecolor{light-gray}{gray}{0.70}

\lstset{
	language=R,
	basicstyle=\footnotesize,
	numbers=left,
	numberstyle=\tiny,
	numbersep=15pt,
	frame=single,
	commentstyle=\color{blue},
	%backgroundcolor=\color{gg} 
}


\begin{document}

\begin{tabular}{c|c|c|c|c|c|c|c|c|c|c}
	   $T$ & $m$* & $N$* & $S_0$  & $K$ & r& $\sigma$ & q & p & n\_estimators & max\_leaf\_nodes\\
	\hline
	              1  & 12 & 10 000 & 100  & 100 &  $1\%$ & $30\% $ & $5\%$ & 7 & 100 & 10
\end{tabular}
\vspace{1cm}

* = When the parameter is not varying

	\begin{figure}[h]
		\includegraphics[scale=0.8]{Rforest_p_assets_n_varying.png}
		\caption{Price with N increasing}
	\end{figure} \hfill
	
	
	\begin{figure}[h]
		\includegraphics[scale=0.8]{Rforest_p_assets_rfestimators_varying.png}
		\caption{Price with n\_estimators increasing}
	\end{figure}
	

	\begin{figure}[h]
		\includegraphics[scale=0.8]{Rforest_7_assets_rf_max_leaf_nodes_varying.png}
		\caption{Price with max\_leaf\_nodes increasing}
	\end{figure}
	
	\begin{figure}[h]
		\includegraphics[scale=0.8]{Rforest_p_assets.png}
		\caption{Price with p increasing}
	\end{figure}
	
	\begin{figure}[h]
		\includegraphics[scale=0.8]{Rforest_p_assets_m_varying.png}
		\caption{Price with m increasing}
	\end{figure}
	

\end{document}